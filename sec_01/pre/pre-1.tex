%Tipo de documento 
	\documentclass[a5paper,oneside]{book}
		
%idioma y compilación

	\usepackage[utf8]{inputenc}
	\usepackage[spanish]{babel}
	\usepackage{graphicx}
	\usepackage{tikz}
	\usepackage{enumerate}
	\usepackage[all]{xy}
	
%Estilos matematicos de la ams

	\usepackage{amsmath,amsfonts,amssymb,amsthm,mathrsfs}
	
%Figuras

	\usepackage{graphicx}
	
%Resaltar amarillo

	\usepackage{pdfpages}
	\usepackage{soul}
	
%Referencias	

	\usepackage{hyperref,enumerate,color}
	
%Notas finales

	\usepackage{endnotes}
	\renewcommand{\notesname}{Notas}
	
%Tamaño de la hoja
	\usepackage[tmargin=20mm,bmargin=20mm,lmargin=15mm,rmargin=15mm]{geometry}
	
%Compilar desde otros archivos.

	\usepackage{subfiles}
	
%Entornos
	\theoremstyle{definition}
	\newtheorem{df}[section]{Definición}

%	\newtheorem{df}[section]{}
	\newtheorem{pr}[section]{Proposición}
	\newtheorem{te}[section]{Teorema}
	\newtheorem{ob}[section]{Observación}
	\newtheorem{op}[section]{Anotación}
	\newtheorem{lm}[section]{Lema}
	\newtheorem{co}[section]{Corolario}
	\newtheorem{nc}[section]{Notación}
	\newtheorem{ej}[section]{Ejemplo}
	\numberwithin{equation}{section}
	\newtheorem{nt}[section]{Nota}
	\newtheorem{rs}[section]{Resumen}
	\newtheorem{ex}[section]{Ejercicio}
	\newtheorem{cn}[section]{Convenio}
	