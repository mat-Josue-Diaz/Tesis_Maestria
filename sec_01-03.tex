% Linea para usar la libreria subfiles.
%\documentclass{subfiles} 

% Préambulo	
%Tipo de documento 
	\documentclass[a5paper,oneside]{book}
		
%idioma y compilación

	\usepackage[utf8]{inputenc}
	\usepackage[spanish]{babel}
	\usepackage{graphicx}
	\usepackage{tikz}
	\usepackage{enumerate}
	\usepackage[all]{xy}
	
%Estilos matematicos de la ams

	\usepackage{amsmath,amsfonts,amssymb,amsthm,mathrsfs}
	
%Figuras

	\usepackage{graphicx}
	
%Resaltar amarillo

	\usepackage{pdfpages}
	\usepackage{soul}
	
%Referencias	

	\usepackage{hyperref,enumerate,color}
	
%Notas finales

	\usepackage{endnotes}
	\renewcommand{\notesname}{Notas}
	
%Tamaño de la hoja
	\usepackage[tmargin=20mm,bmargin=20mm,lmargin=15mm,rmargin=15mm]{geometry}
	
%Compilar desde otros archivos.

	\usepackage{subfiles}
	
%Entornos
	\theoremstyle{definition}
	\newtheorem{df}[section]{Definición}

%	\newtheorem{df}[section]{}
	\newtheorem{pr}[section]{Proposición}
	\newtheorem{te}[section]{Teorema}
	\newtheorem{ob}[section]{Observación}
	\newtheorem{op}[section]{Anotación}
	\newtheorem{lm}[section]{Lema}
	\newtheorem{co}[section]{Corolario}
	\newtheorem{nc}[section]{Notación}
	\newtheorem{ej}[section]{Ejemplo}
	\numberwithin{equation}{section}
	\newtheorem{nt}[section]{Nota}
	\newtheorem{rs}[section]{Resumen}
	\newtheorem{ex}[section]{Ejercicio}
	\newtheorem{cn}[section]{Convenio}
	
	

\begin{document}
	


\section*{Lemas y observaciones}
	
\begin{ob}
	Sea $G$ un grupo topológico.
	
	\begin{enumerate}
	\item Sea $Q(e)$ la componente conexa de la identidad. Como la función multiplicación es continua se sigue que, para toda $h \in G$ el conjunto $hQ(e)$ es conexo y contiene al elemento $h$.

 \item $g \in Q(e)$ si y solo si $e \in Q(g).$ En particular $Q(g)=Q(e)$.
 
 \item  Si $h \in Q(e)$ entonces $e \in Q(h^{-1})$. En particular $h \in Q(e)$ si y solo si $h^{-1}\in Q(e)$.
	\end{enumerate}
 \end{ob}	
 
 \begin{proof}
 Sea $g \in Q(e)$, notemos que $Q(e)$ es un conjunto conexo que tiene a $g$ por tanto $Q(e) \subset Q(g)$, en particular $e \in Q(g).$ El recíproco es idéntico y se omite. Para la otra parte, notemos que $h^{-1}Q(e)$ es un conjunto conexo que tiene a $h^{-1}$ por tanto $h^{-1}Q(e) \subset Q(h^{-1})$, como $h \in Q(e)$ se sigue que el elemento,

$$h^{-1}h \in h^{-1}Q(e)$$

y así $e \in Q(h^{-1})$.  
\end{proof}

\begin{lm}\label{lm:Q(e) es grupo}
La componente conexa de la identidad es un grupo.
\end{lm}


\begin{proof}
 Veremos que el conjunto $Q(e)$ es cerrado bajo la operación de $G$. Sean $g$, $h \in Q(e)$, por la observación previa resta ver que $e \in Q(gh)$. Tenemos que $ ghQ(e) \subset gQ(h) \subset Q(gh)$ por otro lado notemos que, 
 \begin{align*}
 gh(h^{-1}) \in  ghQ(e)
 \end{align*}
 y así $g \in ghQ(e)$, por tanto tenemos la contención $ghQ(e) \subset Q(g)=Q(e)$, concluimos que $gh \in Q(e).$
 \end{proof}	
	
	
	
\begin{lm}\label{lm:gU_es_abierto}
Sea $(X,\tau, \circ)$ grupo topológico y $U$ abierto en $X$, para todo $h \in X$ el conjunto $hU$ es abierto.
\end{lm}

\begin{proof}
Sea $h$ en $X$, las siguientes funciones

\begin{align*}
\varphi_h:X \to X \\
g \mapsto hg
\end{align*}
y 
\begin{align*}
\phi_h:X \to X \\
g \mapsto h^{-1}g
\end{align*}
 son continuas e inversas una de otra. Resta notar que $\varphi_h(U)=hU$ y  la imagen directa de $\varphi$ es la imagen inversa de su función inversa  $\varphi_h(U)=\phi_h^{-1}(U)$ que por continuidad es un conjunto abierto.  Tenemos así que $hU$ es un conjunto abierto.

\end{proof}

% Una vecindad de la identidad genera un grupo conexo. 
\begin{pr} \label{pr:vec_de_la_id_gen}
Sea $(X, \tau, \circ)$ un grupo topológico conexo. Para toda $U(e)$ se cumple que $\langle U\rangle=X.$
\end{pr}
	
\begin{proof}
Sea $U \in \mathcal{N}(e)$, resta ver que $G \subset \langle U \rangle$. Para ello veremos que $\langle U \rangle$ es un conjunto abierto y cerrado en $X$ por la conexidad de $X$ se da la igualdad.

 Sea $g \in \langle U \rangle$, por definición de subgrupo generado, para todo subgrupo $H$ de $X$ que contiene a $U$ se cumple  que 
\begin{enumerate}
	\item $g \in H$,
	\item al ser cerrado como subgrupo tenemos que, para toda $u \in U$ el elemento $gu$ está en $H$, por tanto
 
 $$gU \subset H,$$
 
 \item Por el lema \ref{lm:gU_es_abierto} el conjunto $gU$ es abierto.
\end{enumerate} 
 
 
Más aún, $g=ge \in gU$, de esta manera se que tiene que  $gU(g)$ junto con la contención $gU \subset H$ de la definición de grupo generado tenemos que  $gU \subset \langle U(e) \rangle$ y por tanto $\langle U(e) \rangle$ es un conjunto abierto.

Para ver que $\langle U \rangle$ es cerrado, sea $h \in \overline{\langle U \rangle}$ y consideremos al conjunto $hU$ que, por el lema \ref{lm:gU_es_abierto} es abierto y en particular es vecindad de $h$, $hU(h)$ y por definición del conjunto cerradura tenemos que, 

$$hU \cap \langle U \rangle \neq \emptyset.$$

 De esta manera, sea $g \in hU\cap \langle U \rangle$ ent particular, como $g \in hU$ existe $u \in U$ tal que $g=hu$ y consideremos lo siguiente, 

\begin{align*}
h=gu^{-1} \in  \langle U \rangle U =\langle U \rangle
\end{align*} 
 
 por tanto $\overline{\langle U \rangle} \subset \langle U \rangle.$ Tenemos que $\langle U(e) \rangle$ es un conjunto cerrado a abierto en un espacio conexo, entonces o $\langle U \rangle=\emptyset$ o $\langle U \rangle=X$ como $e \in \langle U \rangle$ concluimos que $\langle U(e) \rangle = X.$
\end{proof}



\begin{df}
Sea $K \subset X$ y $h:X \to X$ un homeomorfismo. Decimos que $h$ está  \textbf{soportado} en $K$ si,
\begin{enumerate}
	\item $X \setminus K$ es no vacío y
	\item $h|_{X \setminus K}=id|_{X \setminus K}$.
\end{enumerate}	
 Al conjunto $$Sup(h)=\overline{\{x \in X : h(x)\neq x \}},$$  le llamaremos el \textbf{soporte} de $h$. Al subgrupo de homeomorfismos, $g:X \to X$, para los cuales existe $U \in \tau$ tal que $g|_U=e|_U$ le denotaremos por $Hom_0(X)$.
\end{df}

\begin{ob} \label{ob:sop_fun_inversa}
Si $g:X \to X$ un homeomorfismo que está soportado en $K$, para la función inversa $g^{-1}:X \to X$ tenemos que 

$$id|_{X \setminus K}=g^{-1}|_{g(X \setminus K)}=g^{-1}|_{X \setminus K}.$$

 Es decir, $g^{-1}$ está soportando en $g(K).$
\end{ob}
El siguiente lema será usado en las secciones posteriores, es importante para el desarrollo del trabajo de Anderson y Epstein en \textbf{referencia}

\begin{lm}\label{lm:obs_a}
 Sean $K \subset X$ y $g \in Hom_0(X)$ soportado en $K$.
 
	\begin{enumerate}
		\item Para cualquier $h \in Hom(X)$ se tiene que $h^{-1}gh$ está soportado en $h^{-1}(K)$.
		\item Si $h^{-1}(K) \cap K = \emptyset$ entonces
			\begin{enumerate}
				\item $[h,g]$ está soportado en $h^{-1}(K) \cup K$,
				\item $[h,g]|_K=g|_K$ y 
				\item $[h,g]|_{h^{-1}(K)}=h^{-1}g^{-1}h|_{h^{-1}(K)}$
			\end{enumerate}	
	\end{enumerate}
\end{lm}
	
\begin{proof}
Para el primer inciso. Sea $x \in X \setminus h^{-1}(K)= h^{-1}( X \setminus K)$ de donde $h(x) \in X \setminus K$, como $K$  es el soporte de $g$ tenemos que,

\begin{align*}
g(h(x))=h(x)
\end{align*}

de esta manera al componer con la función $h^{-1}$ por la izquierda obtenemos lo siguiente,

	\begin{align*}
	h^{-1}gh(x)=x
	\end{align*}

tenemos que
 $$h^{-1}gh(x)|_{X \setminus h^{-1}(K)}=id_{X \setminus h^{-1}(K)}$$ 
 y por definición concluimos que $h^{-1}gh(x)$ está soportado en $h^{-1}(K).$

Ahora veremos el segundo inciso. Supongamos que $h^{-1}(K) \cap K = \emptyset$. Sea $x \in (X \setminus h^{-1}(K)) \cap (X \setminus K)$, como como $g$ está soportando en $K$ tenemos que $$h(g(x))=h(x),$$  más aún como en el inciso anterior tenemos $h(x)\in X \setminus K$ junto con que $g^{-1}|_{X \setminus K}=id$ (observación \ref{ob:sop_fun_inversa}) se sigue que
 
  $$g^{-1}(h(x))=h(x)$$ 
  
 finalmente componiendo con $h^{-1}$ por la izquierda tenemos que  $$ [h^{-1},g^{-1}](x)=h^{-1}g^{-1}hg(x)=x$$
 
 es decir, 
\begin{align*}
 [h^{-1},g^{-1}]|_{(X \setminus h^{-1}(K)) \cap (X \setminus K) }=id,
\end{align*}
 por tanto $[h^{-1},g^{-1}]$ está soportado en $h^{-1}(K) \cup K.$ Finalmente, de la observación \ref{ob:sop_fun_inversa} el homeomorfismo $g^{-1}$ está soportado en $g(K)$ del primer inciso tenemos que $h^{-1}g^{-1}h$ está soportado en $h^{-1}(g(K))$ y de esta manera
	\begin{align*}
	h^{-1}g^{-1}h(h^{-1}(g(K)))=h^{-1}(K),
	\end{align*}	  
  de donde tenemos que
   
\begin{align*}
  [h^{-1},g^{-1}]|_{h^{-1}(K)}=h^{-1}(K).
\end{align*}  
	 Además, notemos $h^{-1}g^{-1}h[g(K)]=id$ entonces  $$g(K) \subset sup(h^{-1}g^{-1}h)^c.$$
\end{proof}



\begin{ob}
Cualquier conjugado del producto de conjugados de producto de $h$ y $h^{-1}$ tiene el mismo numero de conjugados de $h$ y $h^{-1}$.
\end{ob}	 

\begin{proof}
Si $f=(g_1 h^{-1} g_1^{-1}) \cdots(g_n h g_n^{-1})$ entonces para cualquier $g$ en $Hom(X)$ se tiene que
\begin{align*}
g f g^{-1} & = g(g_1 h^{-1} g_1^{-1}) \cdots (g_n h g_n^{-1})g^{-1}\\
&= (gg_1 h^{-1} g_1^{-1}g^{-1})\cdots(gg_n h g_n^{-1}g^{-1}).
\end{align*} 
\end{proof} 

Finalizamos este capitulo recordando que nuestra introducción pudiera no abarcar todos los resultados que vamos a mencionar, sin embargo mencionamos los libros que hemos consultado donde pudiera  estar la demostración detallada  o bien un estudio profundo de dicho tema. \cite{Baz}

\begin{thebibliography}{XX}
%%% Libros de topología
	\bibitem{top_prieto} \textsc{Prieto de Castro, Carlos},
\textit{Topología básica}, segunda edicion,
Ediciones Científicas Universitarias, México, DF, 2013.

	\bibitem{top_salicrup} \textsc{Salicrup Graciela},
\textit{Introducción a la Topología}, Sociedad Matemática Mexicana, México, DF, 1997.

%%% Libros de Grupos

%%% Articulos consultados
	\bibitem{anderson} \textsc{R.D. Anderson}\textit{The Algebraic simplicity of certain groups of homeomorphisms} American Mathematical Society, Enero 1958.
	
	\bibitem{kras} \textsc{J. Krasinkiewicz}\textit{On  homeomorphisms of the Sierpiński curve} Annales societatis Mathematicae Polanae, 1969.
\end{thebibliography}


@article{anderson1958algebraic,
  title={The algebraic simplicity of certain groups of homeomorphisms},
  author={Anderson, Richard D},
  journal={American Journal of Mathematics},
  volume={80},
  number={4},
  pages={955--963},
  year={1958},
  publisher={JSTOR}
}


@book{whyburn1957topological,
  title={Topological characterization of the Sierpinski curve},
  author={Whyburn, Gordon Thomas},
  year={1957},
  publisher={United States Air Force, Office of Scientific Research}
}

@book{whyburn1957topological,
  title={Topological characterization of the Sierpinski curve},
  author={Whyburn, Gordon Thomas},
  year={1957},
  publisher={United States Air Force, Office of Scientific Research}
}

@article{epstein1970simplicity,
  title={The simplicity of certain groups of homeomorphisms},
  author={Epstein, David BA},
  journal={Compositio Mathematica},
  volume={22},
  number={2},
  pages={165--173},
  year={1970}
}

%%%%% Fin del documento
\end{document} 


 \begin{ob}
 \begin{enumerate}
 \item  $f_1gf_1^{-1} \in K$ por la normalidad de $K$.
 \item Como $g^{-1}(U)\cap U= \emptyset$, tenemos que $g(u) \not \in U.$ Por tanto $f^{-1}(g(u))=g(u)$ y así $g^{-1}f^{-1}g(u)=u$ y por tanto $g_1=f_1$ en puntos $u \in U.$
 \end{enumerate}
 \end{ob}

 Luego, consideremos a $f_0=f_2ff_2^{-1}$, salvo homeomorfismo (por $f$ ) $f_0$ tiene soporte $W_0$, luego notemos que si $f_0 \in K$ se sigue que todo conjugado de $f_0$ está en $K$, por tanto $f \in K.$ Resta ver que $f_0 \in K$, consideremos a
 \begin{align*}
 h(x)=\left\lbrace \begin{array}{ccc}
 f_0(x)& \text{ si } & x \in W_0 \\
 g^{n}_1f_0g_1^{-n}(x) & \text{ si }& x \in g_1^{n}(W_0)  n \geq q,\\
x & \text{ si } & x \in X \setminus \cup_{i=0 }^\infty W_i
\end{array}   \right.
\end{align*}

Finalmente, tenemos que $f_0=g_1h^{-1}g_1^{-1}h$.